%\VignetteIndexEntry{A Taxonomy of Estimators}
%\VignetteEngine{R.rsp::tex}
\documentclass[a4paper]{article}
\usepackage{xfrac}
\usepackage{booktabs}
\usepackage[usenames,dvipsnames]{color}
\usepackage[utf8]{inputenc}
\title{A Taxonomy of Estimators}
\author{Andreas Dominik Cullmann}

\begin{document}
\maketitle
\providecommand{\analogy}{--}
\providecommand{\pextended}{\emph{ex}}
\providecommand{\psynthetic}{\emph{sy}}
\providecommand{\psmall}{\emph{sm}}
\providecommand{\rpack}[1]{package \texttt{#1}}
\providecommand{\rdata}[1]{\texttt{#1}}
\providecommand{\rcode}[1]{\texttt{#1}}
\providecommand{\eZ}{\mathcal{Z}}
\providecommand{\strongrule}{\midrule[2pt]}
\providecommand{\lightrule}{\midrule[0.4pt]}
\providecommand{\vR}{\frac{1}{n_{s_2,G}}\hat{V}(\hat{R}_{s_2,G}(x))}
\providecommand{\mR}{\bar{\hat{R}}_{s_2,G}(x)}
\providecommand{\vRc}{\frac{1}{n_{s_2,G}}\hat{V}(\hat{R}_{c,s_2,G}(x))}
\providecommand{\mRc}{\bar{\hat{R}}_{c,s_2,G}(x)}
\section{Introduction}


There is a multitude of estimators given in \cite{Man12a.e},
\cite{Man13a},
\cite{Man13b.e},
\cite{Man13b},
\cite{Man13c.e},
\cite{Man13c} and, finally, 
\cite{hill2017}.

The notation varies, for partially exhaustive auxiliary information, 
the classification given in \cite{hill2017}
even deviates from canon (see \ref{sec:estimators}).

So this is an effort to systematically describe the various small area
estimators.

\paragraph{Superscripts}
For partially exhaustive auxiliary information, 
Mandallaz (\cite[p.~1023]{Man13b}, \cite[p.~383f]{Man13c} defines 
$Z^t(x) = Z^{(1)t}(x)+Z^{(2)t}(x)$ 
whereas Hill \cite[p.~4 and p.~18]{hill2017} defines 
$Z^t(x) = Z^{(0)t}(x)+Z^{(1)t}(x)$.
I will stick with Mandallaz' notation, changing $Z^{(0)t}(x)$ to $Z^{(1)t}(x)$
in Hill's formulae!

\paragraph{Indices}
Mandallaz and Hill inconsistently use the indices $_2$ and $_{s_2}$,
they really both denote the same: the set
$s_2$.
For the sets $s_0$ and $s_1$ they consistently use $_0$ and $_1$.
I change all set indices to $s_{[012]}$.



Hill uses $\bar{Z}_{0,G}^{(1)}$ (and $\bar{Z}_{0}^{(1)}$ which
(\cite[p.~18]{hill2017}) is the exact mean). So I do drop the index,
which is misleadingly referring to some set (and I do so for
$\bar{\eZ}_{0,G}^{(1)}$).


Mandallaz uses $\bar{\hat{R}}_{2,G}$ when calculating the variance of the
residuals in G, for example in \cite[eq 26]{Man13a}, where  $\bar{\hat{R}}_{2,G}$ is clearly
$\bar{\hat{R}}(x)$ while summing over $s_2$ and $G$. I use the latter form.

\paragraph{Residuals}
I have replaced
the empirical mean and variance of the Residuals in G for clustered
sampling,
$$\frac{\sum_{x\in s_2,G}{M(x)\hat{R}_c(x)}}{\sum_{x\in s_2,G}{M(x)}}$$
and
$$\frac{1}{n_{s_2,G}-1} \sum_{x\in
s_2,G}{\left(\frac{M(x)}{\bar{M}(x)}\right)^2 (\hat{R}_c(x)
-\bar{\hat{R}}_c(x))^2},$$
by their shorter notations
$\bar{\hat{R}}_{c,s_2,G}(x)$ and 
$\hat{V}(\hat{R}_{c,s_2,G}(x))$
and likewise for non-clustered sampling.



\section{Estimators\label{sec:estimators}}

In tables~\ref{tab:unclustered2} and \ref{tab:unclustered3},
we see the estimators for the two- and three-phase non-clustered sampling
designs.
The estimators are grouped by the type of auxiliary information: exhaustive (for
 three-phase sampling with full exhaustive auxiliary
information is just two-phase sampling with full exhaustive auxiliary
information with more observations, so there are no estimators),
non-exhaustive and partially exhaustive.
In each block the (pseudo) synthetic the (pseudo) small and the
(pseudo) extended estimator and their variances are given.



Looking at the estimators for partially exhaustive auxiliary information
we see that the estimators and variances are identical for two- and three-phase sampling.
This is due to the fact that 
\cite{hill2017} implement the partially exhaustive auxiliary information using
a full and a reduced model. So they see it as three-phase
sampling where \cite{Man13b} clearly see it as two-phase sampling with partially
exhaustive auxiliary information.



Tables~\ref{tab:clustered2} and \ref{tab:clustered3} give the same information for clustered sampling
designs.



{

    \begin{table}
    \centering
\begin{tabular}{l l l l l r l}
	 exh 	&  type	&  ref & formula\\ \toprule
			
     yes 	&  synthetic 	& \cite[eq. 18]{Man13a} & $\hat{Y}_{G,synth} = \bar{Z}^t_G\hat{\beta}_{s_2}$ \\
        &    	& \cite[eq. 19]{Man13a} & $\hat{V}\left(\times\right) = \bar{Z}^t_G \hat{\Sigma}_{ \hat{\beta}_{s_2}} \bar{Z}_G$ \\
			
\lightrule
     yes 	&  small 	& \cite[eq. 20]{Man13a} & $\hat{Y}_{G,small} = \hat{Y}_{G,synth} +\mR$ \\
            &   	& \cite[eq. 21]{Man13a} & $\hat{V}\left(\times\right) \approx \hat{V}\left(\hat{Y}_{G,synth}\right) + \vR $ \\ 
\lightrule
     yes 	&  extended 	& \cite[eq. 31]{Man13a} & $\hat{\tilde{Y}}_{G,synth} = \bar{\eZ}^t_G\hat{\theta}_{s_2}$ \\
    &    	& \cite[eq. 33]{Man13a} & $\hat{V}\left(\times\right) = \bar{\eZ}^t_G \hat{\Sigma}_{ \hat{\theta}_{s_2}} \bar{\eZ}_G$ \\
\strongrule
			

     no 	&  synthetic 	& \cite[eq. 22]{Man13a} & $\hat{Y}_{G,psynth} = \hat{\bar{Z}}^t_{s_1,G}\hat{\beta}_{s_2}$ \\
        &    	& \cite[eq. 23]{Man13a} & $\hat{V}\left(\times\right) = \hat{\bar{Z}}^t_{s_1,G} \hat{\Sigma}_{ \hat{\beta}_{s_2}} \hat{\bar{Z}}_{s_1,G} + \hat{\beta}^t_{s_2} \hat{\Sigma}_{ \hat{\bar{Z}}_{s_1,G}  } \hat{\beta}_{s_2}$ \\
			
\lightrule
     no 	&  small 	& \cite[eq. 25]{Man13a} & $\hat{Y}_{G,psmall} = \hat{Y}_{G,psynth} + \mR$ \\
            &      	& \cite[eq. 26]{Man13a} & $\hat{V}\left(\times\right) \approx \hat{V}\left(\hat{Y}_{G,synth}\right) + \vR$ \\
\lightrule
     no 	&  extended 	& \cite[eq. 35]{Man13a} & $\hat{\tilde{Y}}_{G,psynth} = \hat{\bar{\eZ}}^t_{s_1,G}\hat{\theta}_{s_2}$ \\
    &    	&  \cite[eq. 36]{Man13a} & $\hat{V}\left(\times\right) = \hat{\bar{\eZ}}^t_{s_1,G} \hat{\Sigma}_{ \hat{\theta}_{s_2}} \hat{\bar{\eZ}}_{s_1,G} + \hat{\theta}^t_{s_2} \hat{\Sigma}_{ \hat{\bar{\eZ}}_{s_1,G}  } \hat{\theta}_{s_2}$ \\
			
\strongrule
			
     part 	&  synthetic 	& \cite[eq. 34]{Man13b} & $\hat{Y}_{psynth,G,greg} = \left( \bar{Z}_G^{(1)}- \hat{\bar{Z}}_{s_1,G}^{(1)}\right) \hat{\alpha}_{s_2} + \hat{\bar{Z}}^t_{s_1,G}\hat{\beta}_{s_2}$\\
        &    	&     \cite[eq. 35]{Man13b} & $\hat{V}\left(\times\right) = \frac{n_{s_2}}{n_{s_1}}\bar{Z}^{(1)t}_G \hat{\Sigma}_{ \hat{\alpha}_{s_2}} \bar{Z}^{(1)}_G + (1-\frac{n_{s_2}}{n_{s_1}})\hat{\bar{Z}}^{t}_{s_1,G} \hat{\Sigma}_{ \hat{\beta}_{s_2}} \hat{\bar{Z}}_{s_1,G} $ \\
\lightrule
			
     part 	&  small 	& \cite[eq. 24]{Man13b} & $\hat{Y}_{G,greg} = \hat{Y}_{psynth,G,greg} + \mR$\\
            &      	& \cite[eq. 23]{Man13b} & $\hat{V}\left(\times\right) \approx \hat{V}\left(\hat{Y}_{psynth,G,greg}\right) + \vR$\\
\lightrule

     part 	&  extended 	& \cite[eq. 30]{Man13b} & $\hat{\tilde{Y}}_{G,greg} = \left( \bar{\eZ}_G^{(1)}- \hat{\bar{\eZ}}_{s_1,G}^{(1)}\right) \hat{\gamma}_{s_2} + \hat{\bar{\eZ}}^t_{s_1,G}\hat{\theta}_{s_2}$\\
        &    	&     \cite[eq. 31]{Man13b} & $\hat{V}\left(\times\right) = \frac{n_{s_2}}{n_{s_1}}\bar{\eZ}^{(1)t}_G \hat{\Sigma}_{ \hat{\gamma}_{s_2}} \bar{\eZ}^{(1)}_G + (1-\frac{n_{s_2}}{n_{s_1}})\hat{\bar{\eZ}}^{t}_{s_1,G} \hat{\Sigma}_{ \hat{\theta}_{s_2}} \hat{\bar{\eZ}}_{s_1,G} $\\
\end{tabular}
\caption{Predictors for non-clustered two-phase sampling, 
    \emph{exh} denotes exhaustiveness of auxiliary information (yes/no/part, the
    latter meaning partially exhaustive auxiliary information),
    \emph{small} denotes the area estimator. \label{tab:unclustered2}
}
\end{table}
\begin{table}
    \centering
\begin{tabular}{ l l r l}
	 exh 	&  type	&  ref & formula\\ \toprule
yes & synthetic & & $-$\\
 &  &  & $-$ \\
\lightrule
yes & small &  & $-$\\
 &  &  & $-$\\
\lightrule
yes & extended &  & $-$\\
 &  &  & $-$\\
\strongrule

     no 	&  synthetic 	& \cite[eq. 26b]{hill2017} & $\hat{Y}_{G,psynth,3p} = \left( \hat{\bar{Z}}_{s_0,G}^{(1)}- \hat{\bar{Z}}_{s_1,G}^{(1)}\right) \hat{\alpha}_{s_2} + \hat{\bar{Z}}^t_{s_1,G}\hat{\beta}_{s_2}$\\
        &    	&    \cite[eq. 26d]{hill2017} & $\hat{V}\left(\times\right) = \hat{\alpha}_{s_2}^t \hat{\Sigma}_{\hat{\bar{Z}}_{s_0,G}^{(1)}} \hat{\alpha}_{s_2} + \frac{n_{s_2}}{n_{s_1}} \hat{\bar{Z}}^{(1)t}_{s_0,G} \hat{\Sigma}_{ \hat{\alpha}_{s_2}} \hat{\bar{Z}}^{(1)}_{s_0,G} + (1-\frac{n_{s_2}}{n_{s_1}})\hat{\bar{Z}}^{t}_{s_1,G} \hat{\Sigma}_{ \hat{\beta}_{s_2}} \hat{\bar{Z}}_{s_1,G} $\\
\lightrule
			
     no 	&  small 	& \cite[eq. 22b]{hill2017} & $\hat{Y}_{G,psmall,3p} = \hat{Y}_{G,psynth,G,3p} + \mR$\\
          &       & \cite[eq. 23b]{hill2017} & $\hat{V}\left(\times\right) \approx \hat{V}\left(\hat{\tilde{Y}}_{G,psynth,3p}\right) + \vR$ \\ 

\lightrule
     no 	&  extended 	& \cite[eq. 23]{Man13c} & $\hat{\tilde{Y}}_{G,g3reg} = \left( \hat{\bar{\eZ}}_{s_0,G}^{(1)}- \hat{\bar{\eZ}}_{s_1,G}^{(1)}\right) \hat{\gamma}_{s_2} + \hat{\bar{\eZ}}^t_{s_1,G}\hat{\theta}_{s_2}$\\
        &    	&     \cite[eq. 24]{Man13c} & $\hat{V}\left(\times\right) = \hat{\gamma}_{s_2}^t \hat{\Sigma}_{\hat{\bar{\eZ}}_{s_0,G}^{(1)}} \hat{\gamma}_{s_2} + \frac{n_{s_2}}{n_{s_1}} \hat{\bar{\eZ}}^{(1)t}_{s_0,G} \hat{\Sigma}_{ \hat{\gamma}_{s_2}} \hat{\bar{\eZ}}^{(1)}_{s_0,G} + (1-\frac{n_{s_2}}{n_{s_1}})\hat{\bar{\eZ}}^{t}_{s_1,G} \hat{\Sigma}_{ \hat{\theta}_{s_2}} \hat{\bar{\eZ}}_{s_1,G} $\\
\strongrule
     part 	&  synthetic 	& \cite[eq. 26a]{hill2017} & $\hat{Y}_{G,synth,3p} = \left( \bar{Z}_{G}^{(1)}- \hat{\bar{Z}}_{s_1,G}^{(1)}\right) \hat{\alpha}_{s_2} + \hat{\bar{Z}}^t_{s_1,G}\hat{\beta}_{s_2}$\\
        &    	&    \cite[eq. 26c]{hill2017} & $\hat{V}\left(\times\right) = \frac{n_{s_2}}{n_{s_1}} \hat{\bar{Z}}^{(1)t}_{s_0,G} \hat{\Sigma}_{ \hat{\alpha}_{s_2}} \hat{\bar{Z}}^{(1)}_{s_0,G} + (1-\frac{n_{s_2}}{n_{s_1}})\hat{\bar{Z}}^{t}_{s_1,G} \hat{\Sigma}_{ \hat{\beta}_{s_2}} \hat{\bar{Z}}_{s_1,G} $\\
\lightrule
			
     part 	&  small 	& \cite[eq. 22a]{hill2017} & $\hat{Y}_{G,small,3p} = \hat{Y}_{G,synth,3p} + \mR$\\
            &      	& \cite[eq. 23a]{hill2017} & $\hat{V}\left(\times\right) \approx \hat{V}\left(\hat{Y}_{G,synth,3p}\right) + \vR$ \\
\lightrule

	 part 	&  extended 	& \analogy{} & $\hat{\tilde{Y}}_{G,ext synth,3p} = \left( \bar{\eZ}_{G}^{(1)}- \hat{\bar{\eZ}}_{s_1,G}^{(1)}\right) \hat{\gamma}_{s_2} + \hat{\bar{\eZ}}^t_{s_1,G}\hat{\theta}_{s_2}$\\
	   	&    	&    \analogy{} & $\hat{V}\left(\times\right) = \frac{n_{s_2}}{n_{s_1}} \bar{\eZ}^{(1)t}_{s_0,G} \hat{\Sigma}_{ \hat{\gamma}_{s_2}} \bar{\eZ}^{(1)}_{s_0,G} + (1-\frac{n_{s_2}}{n_{s_1}})\hat{\bar{\eZ}}^{t}_{s_1,G} \hat{\Sigma}_{ \hat{\theta}_{s_2}} \hat{\bar{\eZ}}_{s_1,G} $\\

\end{tabular}
\caption{Predictors for non-clustered three-phase sampling, 
    \emph{exh} denotes exhaustiveness of auxiliary information (yes/no/part, the
    latter meaning partially exhaustive auxiliary information),
    \emph{small} denotes the area estimator. \label{tab:unclustered3}
}
\end{table}


    \begin{table}
    \centering
\begin{tabular}{l l l l l r l}
	 exh 	&  Type 	&  Reference & Formula\\ \toprule
			
     yes 	&  \psynthetic{} 	& \analogy{} & $\hat{Y}_{c,G,synth} = \bar{Z}^t_G\hat{\beta}_{c,s_2}$ \\
	   	&    	& \analogy{} & $\hat{V} = \bar{Z}^t_G \hat{\Sigma}_{ \hat{\beta}_{s_2}} \bar{Z}_G $\\
\lightrule
			
yes 	&  \psmall{} 	& \analogy{} & $ \hat{Y}_{c,G,small} = \hat{Y}_{c,G,synth} + \mRc$\\
	   	&    	& \analogy{} & $  \hat{V} = \hat{V}\left(\hat{Y}_{c,G,synth}\right) + \vRc$ \\
\lightrule

yes 	&  \pextended{} 	& \cite[eq. 48]{Man13a} & $ \hat{\tilde{Y}}_{c,G,synth} = \bar{\eZ}^t_G\hat{\theta}_{c,s_2}$ \\
    &    	& \cite[eq. 49]{Man13a} & $ \hat{V} = \bar{\eZ}^t_G \hat{\Sigma}_{ \hat{\theta}_{c,s_2}} \bar{\eZ}_G$ \\
			
\strongrule
no 	&  \psynthetic{} 	& \cite[eq. 42]{Man13a} & $ \hat{Y}_{c,G,psynth} = \hat{\bar{Z}}^{1}_{c,s_1,G} \hat{\beta}_{c,s_2}$ \\
        &    	& \cite[eq. 43]{Man13a} & $ \hat{V}(\dot) =\hat{\bar{Z}}^t_{c,s_1,G} \hat{\Sigma}_{ \hat{\beta}_{c,s_2}} \hat{\bar{Z}}_{c,s_1,G} + \hat{\beta}^t_{c,s_2} \hat{\Sigma}_{ \hat{\bar{Z}}_{c,s_1,G}  } \hat{\beta}_{c,s_2} $ \\
\lightrule
			
no 	&  \psmall{} 	& \cite[eq. 44]{Man13a} & $ \hat{Y}_{c,G,psmall} = \hat{Y}_{c,G,psynth} + \mRc$ \\
            &      	& \cite[eq. 45]{Man13a} & $ \hat{V} = \hat{V}\left(\hat{Y}_{c,G,psynth}\right) + \vRc$ \\
\lightrule
no 	&  \pextended{} 	& \cite[eq. 46]{Man13a} & $ \hat{\tilde{Y}}_{c,G,psynth} = \hat{\bar{\eZ}}^t_{c,s_1,G}\hat{\theta}_{c,s_2}$ \\
    &    	&  \cite[eq. 47]{Man13a} & $ \hat{V} = \hat{\bar{\eZ}}^t_{c,s_1,G} \hat{\Sigma}_{ \hat{\theta}_{c,s_2}} \hat{\bar{\eZ}}_{c,s_1,G} + \hat{\theta}^t_{c,s_2} \hat{\Sigma}_{ \hat{\bar{\eZ}}_{c,s_1,G}  } \hat{\theta}_{c,s_2}$ \\
			
			
\strongrule
part 	&  \psynthetic{} 	& \analogy{} & $\hat{Y}_{c,psynth, G,greg} = \left(
\bar{Z}_G^{(1)}- \hat{\bar{Z}}_{c,s_1,G}^{(1)}\right) \hat{\alpha}_{c,2} + \hat{\bar{Z}}^t_{c,s_1,G}\hat{\beta}_{c,s_2}  $\\
	   	&    	&     \analogy{} & $ \hat{V} = \frac{n_{s_2}}{n_{s_1}}\bar{Z}^{(1)t}_G \hat{\Sigma}_{ \hat{\alpha}_{c,s_2}} \bar{Z}^{(1)}_G + (1-\frac{n_{s_2}}{n_{s_1}})\hat{\bar{Z}}^{t}_{c,s_1,G} \hat{\Sigma}_{ \hat{\beta}_{c,s_2}} \hat{\bar{Z}}_{c,s_1,G} $\\
\lightrule
			
part 	&  \psmall{} 	& \analogy{} & $ \hat{Y}_{c,G,greg} = \hat{Y}_{c,G,psynth} + \mRc $\\
	    	&      	& \analogy{} & $ \hat{V}  = \hat{V}\left(\hat{Y}_{c,psynth, G,greg}\right)  + \vRc$\\
\lightrule
part 	&  \pextended{} 	& \cite[eq. 50]{Man13b.e} & $\hat{\tilde{Y}}_{c,G,greg} = \left( \bar{\eZ}_G^{(1)}- \hat{\bar{\eZ}}_{c,s_1,G}^{(1)}\right) \hat{\gamma}_{c,2} + \hat{\bar{\eZ}}^t_{c,s_1,G}\hat{\theta}_{c,2} $\\
        &    	&     \cite[eq. 52]{Man13b.e} &$\hat{V} = \frac{n_{s_2}}{n_{s_1}}\bar{\eZ}^{(1)t}_G \hat{\Sigma}_{ \hat{\gamma}_{c,s_2}} \bar{\eZ}^{(1)}_G + (1-\frac{n_{s_2}}{n_{s_1}})\hat{\bar{\eZ}}^{t}_{c,s_1,G}
\hat{\Sigma}_{ \hat{\theta}_{c,s_2}} \hat{\bar{\eZ}}_{c,s_1,G} $\\
\end{tabular}
\caption{Predictors for clustered two-phase sampling, 
    \emph{exh} denotes exhaustiveness of auxiliary information (yes/no/part, the
    latter meaning partially exhaustive auxiliary information),
    \emph{Type} denotes the area estimator (\psynthetic{} for synthetic, \psmall{}
    for small and \pextended{} for extended synthetic. \emph{Reference} gives the
    refenerce where found, else I derived them by analogy. \label{tab:clustered2}
}
\end{table}
\begin{table}
    \fontsize{8}{2}\selectfont
    \centering
\begin{tabular}{ l l r l}
	 exh 	&  Type	&  Reference & Formula\\ \toprule
     yes & \psynthetic{} & & $-$\\
 &  &  & $-$ \\
\lightrule
yes & \psmall{} &  & $-$\\
 &  &  & $-$\\
\lightrule
yes & \pextended{} &  & $-$\\
 &  &  & $-$\\
\strongrule

no 	&  \psynthetic{} 	& \analogy{} & $\hat{Y}_{G,psynth,3p} = \left( \hat{\bar{Z}}_{c,s_0,G}^{(1)}- \hat{\bar{Z}}_{c,s_1,G}^{(1)}\right) \hat{\alpha}_{c,2} + \hat{\bar{Z}}^t_{c,s_1,G}\hat{\beta}_{c,s_2} $\\
	   	&    	&    \analogy{} & $ \hat{V} = \hat{\alpha}_{c,s_2}^t \hat{\Sigma}_{\hat{\bar{Z}}_{c,s_0,G}^{(1)}} \hat{\alpha}_{c,s_2} + \frac{n_{s_2}}{n_{s_1}} \hat{\bar{Z}}^{(1)t}_{c,s_0,G} \hat{\Sigma}_{ \hat{\alpha}_{c,s_2}} \hat{\bar{Z}}^{(1)}_{c,s_0,G} + (1-\frac{n_{s_2}}{n_{s_1}})\hat{\bar{Z}}^{t}_{c,s_1,G} \hat{\Sigma}_{ \hat{\beta}_{c,s_2}} \hat{\bar{Z}}_{c,s_1,G} $\\
\lightrule
			
no 	&  \psmall{} 	& \analogy{} & $ \hat{Y}_{c,G,psmall,3p} = \hat{Y}_{G,psynth,G,3p} + \mRc$\\
	    	&      	& \analogy{} & $ \hat{V} \approx \hat{V}\left(\hat{\tilde{Y}}_{c,G,psynth,3p}\right) + \vRc$\\
\lightrule

no 	&  \pextended{} 	& \cite[eq. 53]{Man13c.e} & $ \hat{\tilde{Y}}_{c,G,g3reg} = \left( \hat{\bar{\eZ}}_{c,s_0,G}^{(1)}- \hat{\bar{\eZ}}_{c,s_1,G}^{(1)}\right) \hat{\gamma}_{c,2} + \hat{\bar{\eZ}}^t_{c,s_1,G}\hat{\theta}_{c,2}$\\
        &    	&     \cite[eq. 55]{Man13c.e} & $\hat{V} = \hat{\gamma}_{c,s_2}^t \hat{\Sigma}_{\hat{\bar{\eZ}}_{c,s_0,G}^{(1)}} \hat{\gamma}_{c,s_2} + \frac{n_{s_2}}{n_{s_1}} \hat{\bar{\eZ}}^{(1)t}_{c,s_0,G} \hat{\Sigma}_{ \hat{\gamma}_{c,s_2}} \hat{\bar{\eZ}}^{(1)}_{c,s_0,G} + (1-\frac{n_{s_2}}{n_{s_1}})\hat{\bar{\eZ}}^{t}_{c,s_1,G} \hat{\Sigma}_{ \hat{\theta}_{c,s_2}} \hat{\bar{\eZ}}_{c,s_1,G} $\\
\strongrule
part 	&  \psynthetic{} 	& \analogy{} & $\hat{Y}_{c,G,synth,3p} = \left(
\bar{Z}_{G}^{(1)}- \hat{\bar{Z}}_{c,s_1,G}^{(1)}\right) \hat{\alpha}_{c,}2 + \hat{\bar{Z}}^t_{c,s_1,G}\hat{\beta}_{c,s_2} $\\
	   	&    	&    \analogy{} & $ \hat{V} = \frac{n_{s_2}}{n_{s_1}} \hat{\bar{Z}}^{(1)t}_{c,s_0,G} \hat{\Sigma}_{ \hat{\alpha}_{c,s_2}} \hat{\bar{Z}}^{(1)}_{c,s_0,G} + (1-\frac{n_{s_2}}{n_{s_1}})\hat{\bar{Z}}^{t}_{c,s_1,G} \hat{\Sigma}_{ \hat{\beta}_{c,s_2}} \hat{\bar{Z}}_{c,s_1,G} $\\
\lightrule
			
part 	&  \psmall{} 	& \analogy{} & $ \hat{Y}_{c,G,small,3p} = \hat{Y}_{c,G,synth,3p} + \mRc$\\
	    	&      	& \analogy{} & $ \hat{V} \approx \hat{V}\left(\hat{Y}_{c,G,synth,3p}\right) + \vRc$\\
\lightrule

part 	&  \pextended{} 	& \analogy{} & $\hat{\tilde{Y}}_{c,G,ext
synth,3p} = \left( \bar{\eZ}_{G}^{(1)}- \hat{\bar{\eZ}}_{c,s_1,G}^{(1)}\right) \hat{\gamma}_{c,2} + \hat{\bar{\eZ}}^t_{c,s_1,G}\hat{\theta}_{c,2} $\\
	   	&    	&    \analogy{} & $ \hat{V} = \frac{n_{s_2}}{n_{s_1}} \hat{\bar{\eZ}}^{(1)t}_{c,s_0,G} \hat{\Sigma}_{ \hat{\gamma}_{c,s_2}} \hat{\bar{\eZ}}^{(1)}_{c,s_0,G} + (1-\frac{n_{s_2}}{n_{s_1}})\hat{\bar{\eZ}}^{t}_{c,s_1,G} \hat{\Sigma}_{ \hat{\theta}_{c,s_2}} \hat{\bar{\eZ}}_{c,s_1,G}$\\

\end{tabular}
\caption{Predictors for clustered three-phase sampling, 
    \emph{exh} denotes exhaustiveness of auxiliary information (yes/no/part, the
    latter meaning partially exhaustive auxiliary information),
    \emph{Type} denotes the area estimator (\psynthetic{} for synthetic, \psmall{}
    for small and \pextended{} for extended synthetic. \emph{Reference} gives the
    refenerce where found, else I derived them by analogy. \label{tab:clustered3}
}

\end{table}


}


\bibliographystyle{unsrt}
\bibliography{bib}

\end{document}
